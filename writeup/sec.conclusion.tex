\section{Schlussbetrachtung}
Im Rahmen der Arbeit wurde ein Simulationsmodell des Best�ubungsvorgangs von Obstplantagen durch Bienenv�lker aufgestellt und implementiert. Mit Hilfe des Modells wurde die Auswirkung der Bienenstockplatzierung auf die Ausbreitungsgeschwindkeit des ABPV untersucht. Aus den Ergebnissen der Experimente l�sst sich schlussfolgern, dass eine Platzierungsstrategie ausgew�hlt werden sollte, bei der sich die einzelnen Bienenst�cke m�glichst weit auseinander befinden.

Nach einem Wechsel der Simulationsframeworks auf ein Performanteres k�nnten die Experimente mit gr��erer Bienenanzahl und mehr Wiederholungen erneut durchgef�hrt und so die Konfidenz der Ergebnisse erh�ht werden.

Offen ist die Frage, wie sich das System bei nicht betrachteten Konfigurationen verh�lt. Es k�nnten beispielsweise nur bestimmte St�cke oder nur eine einzelne Biene von Beginn an krank sein. Au�erdem wurde die Auswirkung der genetischen Diversit�t durch Mutationen und Immunisierung au�er Acht gelassen. Die verwendeten Platzierungsstrategien sind relativ beschr�nkt und eine Untersuchung anderer Platzierungen k�nnte weitere Ergebnisse liefern.

Ein weiterer interessanter Untersuchungsgegenstand ist die Frage, ab welcher Stelle eine Intervention durch den Imker das Kollabieren der St�cke gerade noch verhindern kann. So k�nnte eine Handlungsempfehlung f�r das Verabreichen von Medikamenten erteilt werden. Hierf�r ist jedoch weitere Forschungsarbeit notwendig.

