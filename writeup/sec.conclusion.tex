\section{Schlussbetrachtung}

Zusammenfassung
\begin{itemize}
\item
\end{itemize}

Kritische W�rdigung
\begin{itemize}
\item Threadproblematik (und damit verbundene Limitation bei den Experimenten -> Zeit und Entit�tszahl))
\item Andere Einfl�sse wie Medikamentation oder genetische Diversit�t / Immunisierung unber�cksichtigt
\item Konfigurationen mit Strategie nach fest vorgegebenem Schema
\end{itemize}

Das benutzte Framework erzeugt f�r jede Entit�t einen neuen Thread. Die Java Virtual Machine reserviert f�r jeden Thread eine bestimmte Speichermenge. So kommt bei ausreichend gro�e Anzahl von Entit�ten dazu, dass zu wenig Arbeitsspeicher vorhanden ist. Deshalb werden Teile auf die Festplatte ausgelagert, wodurch sich die Dauer einer Simulation stark vergr��ert. \textbf{Gibt es Medizin speziell gegen ABPV??? Betty?}. Ein Imker k�nnte nach dem Bemerken einer Erkrankung seiner Bienen ein Medikament verabreichen, um einer weiteren Ausbreitung vorzubeugen. In der Natur besitzen manche Bienenst�mme eine nat�rliche Immunit�t gegen das Virus oder k�nnen diese nach einer Infektion entwickeln. Diese Aspekte wurden in diesem Projekt nicht betrachtet. So k�nnte nach einer Ansteckung noch entschieden werden, ob eine Biene auch erkrankt. Desweiteren w�re es m�glich die Bienen einzelner Bienenst�cke immun gegen die Krankheit zu machen. Bisher werden die Bienenst�cke nach einem festen Schema platziert: Es gibt Gruppen von St�cken, die mit gr��tm�glicher Entfernung untereinander platziert werden. Innerhalb einer Gruppe werden die St�cke nah nebeneinander gesetzt. Dabei werden andere Anordnungen nicht simuliert. So sollten auch weitere geometrische oder zuf�llig gew�hlte Platzierungen �berpr�ft werden.

Ausblick / Erweiterungsm�glichkeiten / \textbf{oben mit abhandeln?}
\begin{itemize}
\item Einbezug von Immunisierung oder genetischer Diversit�t
\item genauere Untersuchung des Infektionsverlaufs (nicht nur Zeit bis alle tot sind sondern auch lokale Ausbreitungsgeschwindigkeit z.B.)
\end{itemize}



