\section{Schlussbetrachtung}
Aus Ergebnissen des Experiments l�sst sich schlussfolgern, dass tendenziell eine Platzierungsstrategie ausgew�hlt werden sollte, bei der sich die einzelnen Bienenst�cke m�glichst weiter auseinander befinden. Ob sich dieses Ergebnis der Simulation auf die Realit�t �bertragen l�sst, ist schwer zu sagen. F�r das Simulationsmodell mussten verschiedene Annahmen getroffen werden bzw. war eine naturgetreue Abbildung aus technischen Gr�nden nicht m�glich.

Das genutzte Framework erzeugt f�r jede Entit�t einen neuen Thread, f�r welchen die Java Virtual Machine einen bestimmten Teil der Systemressourcen reserviert. So kommt es bei ausreichend gro�er Anzahl von Entit�ten dazu, dass die Ressourcen �berlastet werden. Daher kann sich die Dauer eines Simulationslaufes bei entsprechend gro�er Anzahl von Entit�ten stark vergr��ern.

Ein Imker k�nnte nach dem Bemerken einer Erkrankung seiner Bienen ein Medikament verabreichen, um einer weiteren Ausbreitung vorzubeugen. In der Natur besitzen einige Bienenst�mme eine nat�rliche Immunit�t gegen das Virus oder k�nnen diese nach einer Infektion entwickeln. Sowohl die Verabreichung von Medikamenten, als auch eine eventuell vorhandene Immunit�t wurden in diesem Projekt nicht betrachtet. So k�nnte nach einer Ansteckung noch entschieden werden, ob eine Biene auch tats�chlich erkrankt. Desweiteren w�re es m�glich die Bienen einzelner Bienenst�cke gegen die Krankheit zu immunisieren.

Bisher werden die Bienenst�cke nach einem festen Schema platziert: Es gibt Gruppen von St�cken, die mit gr��tm�glicher Entfernung untereinander platziert werden. Innerhalb einer Gruppe werden die St�cke nah nebeneinander gesetzt. Dabei werden andere Anordnungen nicht simuliert. So sollten auch weitere geometrische Formen oder zuf�llig gew�hlte Platzierungen �berpr�ft werden.

Als weiterer Indikator k�nnte die Ausbreitungsgeschwindigkeit der Krankheit innerhalb eines Stocks dienen. So k�nnte ein Zeitpunkt ermittelt werden, der zwangsl�ufig zum Absterben des Stocks f�hrt. M�glicherweise w�re das der sp�teste Zeitpunkt zur effektiven Medikamentenverabreichung durch den Imker.
