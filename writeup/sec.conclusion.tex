\section{Schlussbetrachtung}

Zusammenfassung
\begin{itemize}
\item
\end{itemize}

Kritische W�rdigung
\begin{itemize}
\item Threadproblematik (und damit verbundene Limitation bei den Experimenten -> Zeit und Entit�tszahl))
\item Andere Einfl�sse wie Medikamentation oder genetische Diversit�t / Immunisierung unber�cksichtigt
\item Konfigurationen mit Strategie nach fest vorgegebenem Schema
\end{itemize}

Das genutzte Framework erzeugt f�r jede Entit�t einen neuen Thread, welchem die Java Virtual Machine einen bestimmten Teil des Arbeitsspeichers reserviert. So kommt es bei ausreichend gro�er Anzahl von Entit�ten dazu, dass zu wenig Arbeitsspeicher vorhanden ist. Daher werden Teile auf die Festplatte ausgelagert, wodurch sich die Dauer einer Simulation stark vergr��ert. 

Ein Imker k�nnte nach dem Bemerken einer Erkrankung seiner Bienen ein Medikament verabreichen, um einer weiteren Ausbreitung vorzubeugen. In der Natur besitzen einige Bienenst�mme eine nat�rliche Immunit�t gegen das Virus oder k�nnen diese nach einer Infektion entwickeln. Sowohl die Verabreichung von Medikamenten, sowie eine eventuell vorhandene Immunit�t wurden in diesem Projekt nicht betrachtet. So k�nnte nach einer Ansteckung noch entschieden werden, ob eine Biene auch erkrankt. Desweiteren w�re es m�glich die Bienen einzelner Bienenst�cke gegen die Krankheit zu immunisieren. Bisher werden die Bienenst�cke nach einem festen Schema platziert: Es gibt Gruppen von St�cken, die mit gr��tm�glicher Entfernung untereinander platziert werden. Innerhalb einer Gruppe werden die St�cke nah nebeneinander gesetzt. Dabei werden andere Anordnungen nicht simuliert. So sollten auch weitere geometrische oder zuf�llig gew�hlte Platzierungen �berpr�ft werden. Als weiterer Indikator k�nnte die Ausbreitungsgeschwindigkeit der Krankheit innerhalb eines Stocks beobachtet werden. So k�nnte ein Zeitpunkt ermittelt werden, der zwangsl�ufig zum Absterben des Stocks f�hrt. M�glicherweise w�re das der sp�teste Zeitpunkt zur effektiven Medikamentenverabreichung durch den Imker.

Ausblick / Erweiterungsm�glichkeiten / \textbf{oben mit abhandeln?}
\begin{itemize}
\item Einbezug von Immunisierung oder genetischer Diversit�t
\item genauere Untersuchung des Infektionsverlaufs (nicht nur Zeit bis alle tot sind sondern auch lokale Ausbreitungsgeschwindigkeit z.B.)
\end{itemize}



