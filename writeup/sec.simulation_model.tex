\section{Simulationsmodell}

Um die Best�ubung seiner der Ackerfl�chen sicherzustellen, erteilt ein Bauer einem Imker einen Best�ubungsauftrag. Dem Imker steht die Ackerfl�che zur Verf�gung um seine Bienenst�cke zu platzieren, damit die Best�ubung optimal stattfinden kann. In der Simulation wirken der Imker, die Bienenst�cke mit Bienenpopulationen, die Krankheit und die Blumen zusammen.

Innerhalb eines Stocks gibt es eine K�nigin, Arbeiter und Drohnen. Jeder Bienenstock hat eine K�nigin, die solange neue Eier legt, bis die Kapazit�t des eigenen Stocks erreicht ist. Ein Arbeiter fliegt zu Blumen, um eine Menge von Nektar zu sammeln und diesen zur�ck in den Bienenstock zu bringen. Nach einem Sammelvorgang ruht sich die Biene aus. Arbeiter k�nnen unbeabsichtigt zu einem fremden Stock fliegen. Dort geben sie den Nektar ab und werden danach ausgesto�en. Eine Biene stirbt nach einer bestimmten Zeit eines nat�rlichen Todes oder verendet durch die Krankheit. Ein Bienenstock kollabiert, falls eine bestimmte Anzahl lebender Bienen unterschritten wurde.

Bienen k�nnen sich mit der Krankheit infizieren. Falls sich eine Biene angesteckt hat, kann sie die Krankheit auf weitere Bienen �bertragen. Eine �bertragung kann entweder innerhalb eines Stocks oder beim Zusammentreffen zweier Bienen an einer Bl�te geschehen. Zwischen der Infizierung einer Biene mit der Krankheit und dem Ausbruch liegt eine bestimmte Zeit.

Die Blumen sind gleichm��ig �ber die Fl�che verteilt. Jede Blume verf�gt �ber eine Menge von Nektar. Nachdem der Nektar entnommen wurde, dauert es eine bestimmte Zeit bis sich der Vorrat wieder auff�llt.

Das Ziel der Simulation ist herauszufinden, ob die Platzierung der Bienenst�cke Einfluss auf die Zeitdauer bis zum Kollaps aller Bienenpopulationen eines Imkers hat. Dazu soll die Zeitdauer bis zum Kollaps unter verschiedenen Platzierungsszenarien gemessen werden.

Zur Vereinfachung des Modells wurden bestimmte Annahmen getroffen. So entf�llt die explizite Unterscheidung zwischen Drohnen und Arbeitern. Es wird davon ausgegangen, dass ein bestimmter Prozentsatz der Bienen sich immer innerhalb des Stocks aufh�lt. Um einen Krankheitsausbruch abzubilden, ist am Anfang eines Simulationslaufs ein bestimmter Teil der Bienen erkrankt.

Um eine Verzerrung der Outputgr��en zu unterbinden, werden alle Bienenst�cke zu Beginn einer Simulation mit einer zuf�lligen Bienenpopulation ausgestattet. D.h. in jedem Stock ist bereits eine Anzahl von Bienen vorhanden, die sich in unterschiedlichen Lebensabschnitten befinden.

Die Simulation ist dynamisch und ereignisdiskret. Das System wird �ber einen bestimmten Zeitraum betrachtet und der Zustand des Systems �ndert sich durch das Eintreten bestimmter Ereignisse. Solche Ereignisse w�ren beispielsweise: das Schl�pfen und Sterben einer Biene, das Losfliegen zu einer Bl�te oder die Ansteckung einer Biene mit der Krankheit. Als Inputdaten werden teils Daten aus wissenschaftlichen Ver�ffentlichungen, teils durch Sensitivit�tsanalysen ermittelte Daten benutzt. Die Simulation endet, wenn alle Bienenpopulationen kollabiert sind.
