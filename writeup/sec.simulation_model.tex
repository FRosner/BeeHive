\section{Simulationsmodell}

In diesem Projekt wurde nur ein Ausschnitt des Problems betrachtet. Daher folgt nun eine Abgrenzung des Simulationsmodells. Ein Imker befinde sich in folgender Situation: Ihm steht eine Anbaufl�che zur Verf�gung, auf der eine bestimmte Anzahl von Bienenst�cken zu platzieren ist. So gibt es verschiedene Gr��en, die an der Simulation beteiligt sind: der Imker, Bienenst�cke mit Bienenpopulationen, die Krankheit und die Blumen.

Innerhalb eines Stocks gibt es eine K�nigin, Arbeiter und Drohnen. Jeder Bienenstock hat eine K�nigin, die solange neue Eier legt, bis die Kapazit�t des eigenen Stocks erreicht ist. Ein Arbeiter fliegt zu Blumen um eine Menge von Nektar zu sammeln und diesen zur�ck in den Bienenstock zu bringen. Nach einem Sammelvorgang ruht sich die Biene aus. Arbeiter k�nnen unabsichtlich zu einem fremden Stock fliegen. Dort geben sie den Nektar ab und werden danach ausgesto�en. Eine Biene stirbt nach einer bestimmten Zeit eines nat�rlichen Todes oder verendet durch die Krankheit. Ein Bienenstock kollabiert, falls eine bestimmte Anzahl lebender Bienen unterschritten wurde.

Bienen k�nnen sich mit der Krankheit infizieren. Falls sich eine Biene angesteckt hat, kann sie die Krankheit auf weitere Bienen �bertragen. Eine �bertragung kann entweder innerhalb eines Stocks oder beim Zusammentreffen zweier Bienen an einer Bl�te geschehen. Zwischen der Infizierung einer Biene mit der Krankheit und dem Ausbruch liegt eine bestimmte Zeit. Einzelne Bienen stecken sich nicht mit der Krankheit an, weil sie eine nat�rliche Immunit�t besitzen.

Die Blumen sind auf eine bestimmten Weise angeordnet. Jede Blume verf�gt �ber eine Menge von Nektar. Nachdem der Nektar entnommen wurde, dauert es eine bestimmte Zeit bis sich der Vorrat wieder auff�llt.

Das Ziel der Simulation ist herauszufinden, ob die Platzierung der Bienenst�cke Einfluss auf die Zeitdauer bis zum Kollaps aller Bienenpopulationen eines Imkers hat. Dazu soll die Zeitdauer bis zum Kollaps unter verschiedenen Platzierungsszenarien gemessen werden.

Die Simulation ist dynamisch und ereignisdiskret. Das System wird �ber einen bestimmten Zeitraum betrachtet und der Zustand des Systems �ndert sich durch das Eintreten bestimmter Ereignisse. Solche Ereignisse w�ren bspw.: das Schl�pfen und Sterben einer Biene, das Losfliegen zu einer Bl�te oder die Ansteckung einer Biene mit der Krankheit. Als Inputdaten werden teils Daten aus wissenschaftlichen Ver�ffentlichungen, teils durch Sensitivit�tsanalysen ermittelte Daten benutzt. Die Simulation endet, wenn alle Bienenpopulationen zusammengebrochen sind.