\section{Einleitung}

Das Finden von korrelierten W�rtern in gro�en Dokumentsammlungen ist eine wichtige Zutat f�r das Analysieren von Texten. Kommen zwei W�rter h�ufig gemeinsam in einzelnen Dokumenten einer Sammlung vor, so weisen sie eine hohe Korrelation auf. Solche korrelierten W�rter k�nnen verwendet werden, um automatisiert Thesauri zu erzeugen (\cite{baeza1992introduction,lin1998automatic,lassi2002automatic}). Das Finden von Synonymen ist ebenfalls eine wichtige Anwendung des Textminings (\cite{turney2001mining}).

Zur Bestimmung hoher Wortkorrelationen einer Dokumentsammlung kann eine Korrelationsmatrix mit den paarweisen Korrelationen aller W�rter berechnet werden, aus der anschlie�end die hohen Korrelationen extrahiert werden. Beinhalten Dokumentsammlungen viele Dokumente und W�rter, ist dies jedoch auf Grund des quadratischen Rechenaufwandes kaum noch m�glich. Um dennoch in der Lage zu sein, hohe und damit interessante Wortkorrelationen zu bestimmen, l�sst sich eine von \textcite{indyk1998approximate} entwickelte Technik zum Finden �hnlicher Objekte im hochdimensionalen Raum, das sogenannte Locality-Sensitive-Hashing (LSH), anwenden. \textcite{charikar2002similarity} hat ein spezielles LSH-Schema entwickelt, welches das effiziente Finden von Vektoren mit gro�er Kosinus�hnlichkeit in einem hochdimensionalen Vektorraum erm�glicht. Werden zwei Vektoren mit Hilfe dieses Schemas gehasht, haben sie eine um so gr��ere Kollisionswahrscheinlichkeit, je kleiner der von ihnen eingeschlossene Winkel ist. Werden W�rter als Vektoren repr�sentiert, ist es m�glich, �hnliche W�rter schnell herauszufinden. Werden die Wortvektoren zentriert, so ist die Kollisionswahrscheinlichkeit proportional zur Korrelation. Somit ist es nicht n�tig alle m�glichen Korrelationen auszurechnen. Es gen�gt, alle W�rter einmalig zu hashen und Kollisionen mit Hilfe einer Sortierung der Hashwerte herauszufiltern. Kollisionen treten zwischen Vektoren potenziell hoch korrelierter W�rter auf.

Das Finden �hnlicher Objekte in hochdimensionalen Vektorr�umen ist ein aktuelles Forschungsthema. \textcite{Zhai:2011:APA:1989323.1989428} untersuchten das effiziente Finden �hnlicher Vektoren mit Hilfe eines probabilistischen Similarity-Search-Algorithmus. \textcite{Bayardo:2007:SUP:1242572.1242591} entwickelten einen auf Neuindizierung und Optimierungsstrategien basierenden Algorithmus f�r dieses Problem. Weitere Ans�tze wurden unter anderem von \textcite{zhu2011scaling} und \textcite{Awekar:2009:IPS:1731011.1731012} untersucht. Ein Vergleich von LSH gegen einen Brute-Force-Ansatz zum Extrahieren �hnlicher Dokumente �ber verschiedene Sprachen hinweg wurde von \textcite{Ture:2011:NFL:2009916.2010042} vorgenommen.

Ziel dieser Arbeit ist es, ein effizientes Verfahren zum Finden hoher Wortkorrelationen mittels LSH vorzustellen und zu evaluieren. Im ersten Abschnitt der Arbeit wird zun�chst erl�utert, wie gro�e Dokumentsammlungen modelliert werden. Es wird eine formale Definition von Wortkorrelationen aufgestellt und diese im Kontext des Modells beleuchtet. Anschlie�end folgt die Vorstellung des naiven Algorithmus sowie des effizienten Algorithmus mittels LSH zur Bestimmung hoher Wortkorrelationen. Im dritten Abschnitt schlie�en sich Experimente zur Evaluation der effizienten Methode an. Abschlie�end folgt eine Zusammenfassung, eine kritische W�rdigung und ein Ausblick auf weitere Forschungsarbeit. Die Ergebnisse der Arbeit wurden durch Implementierung der Algorithmen sowie den Entwurf und die Durchf�hrung zahlreicher Experimente ermittelt. F�r die Implementierung der Algorithmen wurde die Programmiersprache \texttt{R} verwendet. Die Grafiken wurden mit Hilfe von \texttt{gnuplot}, \texttt{R} und \texttt{tikz} erstellt. Die Vorverarbeitung der Datens�tze zur experimentellen Evaluation geschah mit Hilfe selbst geschriebener \texttt{Java}-Programme und \texttt{Apache} \texttt{Lucene}. Die Zusammenarbeit der einzelnen Programme wurde durch \texttt{Shell} \texttt{Scripts} koordiniert.
