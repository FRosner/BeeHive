\section{Einleitung}

Biodiversit�tsverlust und "`biologische Verschmutzung"' sind gro�e Probleme der modernen Zeit (\cite{vorosmarty2010global, daszak2000emerging, vitousek1997human, fidler1996globalization}). Auch bei Bienen gibt es in den letzten Jahren Anlass zur Sorge. Bienen sind f�r die Best�ubung vieler Obst- und Gem�sesorten verantwortlich und stellen einen wichtigen Teil des �kosystems dar (\cite{dewenter2005pollinators}). \textcite{aizen2009global} stellten fest, dass die Anzahl der verwalteten Bienenkolonien in Europa und Nordamerika stark zur�ckgegangen ist. Ursachen hierf�r sind neben �konomischen, biologischen und Umweltfaktoren auch eine Reihe von Krankheiten und Parasiten (\cite{genersch2010honey}). \textcite{genersch2010german} identifizierten die Varroamilbe, den Deformed Wing Virus (DWV) und den Acute Bee Paralysis Virus (ABPV) als Ursachen f�r das Bienensterben in Deutschland.

Im Rahmen des vorliegenden Simulationsprojektes soll die Ausbreitung einer Krankheit in Bienenpopulationen w�hrend der Best�ubung einer Obstplantage untersucht werden. Interessant ist die Fragestellung, welche Faktoren die Ausbreitung der Krankheit beeinflussen. Um die Komplexit�t des Projektes angemessen zu halten, beschr�nken sich die Untersuchungen auf den ABPV\footnote{Weitere Informationen zum ABPV finden sich in \textcite{genersch2010emerging} und \textcite{ages2007abpv}.}. Als Einflussfaktor soll die Anordnung der Bienenst�cke auf der Plantage n�her beleuchtet werden. Zus�tzlich werden zahlreiche Modellannahmen getroffen.
