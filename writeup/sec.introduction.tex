\section{Einleitung}

Biodiversit�tsverlust und "`biologische Verschmutzung"' sind gro�e Probleme der modernen Zeit (\cite{fidler1996globalization, vitousek1997human, daszak2000emerging, vorosmarty2010global}). Auch das pl�tzliche Sterben von Bienenv�lkern gibt in den letzten Jahren Anlass zur Sorge. Bienen sind f�r die Best�ubung vieler Obst- und Gem�sesorten verantwortlich und stellen damit einen wichtigen Teil des �kosystems dar (\cite{dewenter2005pollinators}).

\textcite{aizen2009global} stellten fest, dass die Anzahl der verwalteten Bienenkolonien in Europa und Nordamerika stark zur�ckgegangen ist. Ursachen hierf�r sind neben �konomischen, biologischen und Umweltfaktoren auch eine Reihe von Krankheiten und Parasiten (\cite{genersch2010honey}). \textcite{genersch2010german} identifizierten die Varroamilbe, das Deformed Wing Virus (DWV) und das Acute Bee Paralysis Virus (ABPV) als Ursachen f�r das Bienensterben in Deutschland.

Im Rahmen des vorliegenden Simulationsprojektes soll die Ausbreitung einer Krankheit in Bienenpopulationen w�hrend der Best�ubung einer Obstplantage untersucht werden. Interessant ist die Fragestellung, welche Faktoren die Ausbreitung der Krankheit beeinflussen. Um die Komplexit�t des Projektes angemessen zu halten, beschr�nken sich die Untersuchungen auf den ABPV\footnote{Weitere Informationen zum ABPV, dem Krankheitsbild und der Auswirkung auf Bienenv�lker finden sich in \textcite{ages2007abpv} und \textcite{genersch2010emerging}.}. Als Einflussfaktor soll die Anordnung der Bienenst�cke auf der Plantage n�her betrachtet werden. Zus�tzlich werden verschiedene Modellannahmen getroffen.

Ziel des Projekts ist es, den Einfluss der Bienenstockpositionen w�hrend der Best�ubung einer Obstplantage auf die Ausbreitungsgeschwindigkeit des ABPV zu untersuchen. Hierf�r wird ein Simulationsmodell erstellt, implementiert und verschiedene Szenarien simuliert und verglichen. Im ersten Abschnitt der Arbeit werden das Simulationsmodell und die getroffenen Annahmen beschrieben. Anschlie�end folgt eine Erl�uterung der Implementierung des Modells in \emph{Java}.\footnote{Der entstandene Quellcode und die dazugeh�rige Dokumentation kann unter \\ https://github.com/FRosner/BeeHive eingesehen werden.} Der darauffolgende Abschnitt widmet sich der Versuchsplanung, den Experimenten und Ergebnissen. Abschlie�end findet sich eine Zusammenfassung, eine kritische W�rdigung und eine Diskussion weiterer Untersuchungsans�tze.
